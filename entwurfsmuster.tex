\section{Entwurfsmuster}
\label{sec:entwurfsmuster}

\subsection{Factory}
\label{subsec:factory}

Im Projekt lässt sich relativ einfach ein konkretes \emph{Entwurfsmuster} identifizieren.
Hierbei handelt es sich um das \emph{Factory} \emph{Entwurfsmuster} in der für die DSL Sprache verwendeten \texttt{Builder} Klassen.

Sie erzeugen jeweils Datenobjekte spezifisch zu ihrem Zweck.
Der \texttt{ContainerBuilder} (siehe \Cref{lst:container_builder}) beispielsweise konsturiert \texttt{Container} Objekte und befüllt diese mit den vom Benutzer bereitgestellten Informationen.
Anschließend lässt sich das Objekt mit einer Methode zurückgeben.
Im Falle des \texttt{ContainerBuilder} kann der \texttt{Container} mit der Methode \texttt{toContainer()} erzeugt werden.

Die \texttt{Builder} geben darüberhinaus die Grammatik der internen DSL an und sind ein klassisches Entwurfsmuster für Kotlin DSL Sprachen.
Das erfolgt durch eine Verzahnung der einzelnen \texttt{Builder} miteinander.
Für den \texttt{Container} ist es zum Beispiel erforderlich das \texttt{Image} anzugeben.
Hierfür existiert dann ein weiterer \texttt{Builder}, den der Benutzer innerhalb des \texttt{ContainerBuilder} mit einer vorgegeben Methode aufrufen muss.

\subsection{Decorator}
\label{subsec:decorator}

Die DSL Sprache besitzt an einigen Stellen die Möglichkeit eine sogenannte \emph{Short Hand} Syntax zu nutzen.
Sie dient der vereinfachten Nutzung innerhalb eines \texttt{Builder}.
Ein Beispiel für eine solche \emph{Short Hand} Syntax findet sich im \Cref{lst:mount_builder} des \texttt{MountBuilder}.
Zur Umsetzung dieser Syntax werden entsprechende Objekte konstruiert, deren Informationen an weitere Objekte nach der Verwendung eines Infix-Operators weitergeben werden.
Bisher sind diese Objekte noch unabhängig voneinander.

An dieser Stelle lässt sich zukünftig auf das \emph{Decorator} Entwurfsmuster zurückgreifen.
Dann würde ein Basisobjekt existieren, von dem weitere \enquote{dekorierte} Objekte abgeleitet wären.
So ergäbe sich eine wohldefinierte Abhängigkeitskette für den Aufbau dieser für die \emph{Short Hand} Syntax verwendeten Typen.
