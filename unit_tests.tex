\section{Unit tests}
\label{sec:unit_tests}

Für das Projekt wurden insgesamt 29 \emph{Unit Tests} implementiert, die allesamt bisher ihren Bezug auf die von der Anwendung bereitgestellte DSL Sprache legen.
Sie finden sich im \texttt{test} Modul des Projekts und sind entsprechende ihrer zu testenden Klassen in Pakete gegliedert.
Andere Teile der Anwendung müssen zukünftig noch mit Tests ausgestattet werden, um auch deren korrektes Verhalten automatisiert zu verifizieren.

Aus den Tests resultiert eine Line Coverage von 49\% insgesamt für das Projekt.
Spezifisch für das Paket \texttt{it.oechsler.script}, für das die Tests bisher bereitstehen, ist die resultierende Coverage mit 62\% etwas höher.

Zur Umsetzung der \emph{Unit Tests} wurde auf das \emph{Framework} \texttt{kotest}\footnote{\url{https://kotest.io}} zurückgegriffen.
Es unterstützt verschiedene Stile aus der Welt des automatisierten Testens.
Spezifisch verwendet wurde der \texttt{DescribeSpec} Stil, welcher vor allem in den \emph{JavaScript} \emph{Test Frameworks}, wie etwa \emph{Jest}\footnote{\url{https://jestjs.io}}, Anwendung findet.

Eine Ausführung der \emph{Unit Tests} wird automatisch beim Erzeugen des \emph{Docker} \emph{Image} angestoßen, kann aber auch mit \emph{Gradle} manuell ausgeführt werden.
