\section{Clean Architecture}
\label{sec:clean_architecture}

In diesem Abschnitt sollen die für \emph{Clean Architecture} in diesem Projekt relevanten Schichten definiert werden.
Diese Schichten sind konkret: \emph{Domänen-}, \emph{Applikations-}, \emph{Adapter-} und \emph{Plugin-Schicht}.

Zusätzlich existiert noch eine \emph{Abstraktions-Code-Schicht}.
Sie umfasst Grundbausteine, die nicht in dem Bereich der Domäne zugehörig sind.
Es handelt sich also um sehr abstrakte Konzepte und Algorithmen (etwa Sortieralgorithmus, oder Datenstrukturen).
Derartige Grundlagen wurden allerdings in dieser Anwendung quasi nicht implementiert, daher wird dieser Bereich in den folgenden Sektionen nicht weiter berücksichtigt und stattdessen direkt auf ihn eingegangen.

Allerdings gibt es in der Anwendung eine Impementierung von \emph{Speichereinheiten}.
Diese könnten in die Kategorie \emph{Abstraltions-Code} fallen.
Konkret gemeint ist die Klasse \texttt{Size} aus dem \texttt{it.oechsler.script.data} Paket.
Sie definiert eine Quantität an Speicher.
Diese Quantität wird in \emph{Gigabyte} normiert.
Bedeutet Konkret: Ist eine Quantität von eins vergben, so entspricht dies einem \emph{Gigabyte}.
Für Konvertierung zwischen verschiedenen Speichereinheiten existieren daher als Zusatz \emph{Extension Methods} in \texttt{SizeExtensions.kt} im \texttt{it.oechsler.script.extensions} Paket, die es erlauben gängige primitivtypen möglichst einfach und Korrekt in eine \texttt{Size} Quantität zu wandeln.

Da \emph{Speichereinheiten} sich nicht verändernde Maßeinheiten darstellen, ist es möglich diesen Bereich der Anwendung zum \emph{Abstraktions-Code} zu zählen.

\subsection{Domänen-Schicht}
\label{subsec:domaenen_schicht}

\begin{figure}[ht]
    \centering

    \includeplantuml[width=\textwidth]{assets/domain}

    \caption{Klassendiagramm der \emph{Domänen-Schicht}}
    \label{fig:domain}
\end{figure}

Bei der \emph{Domänen-Schicht}, oder dem \emph{Domänen-Code} handelt es sich um \emph{Entitäten}, welche die grundlegende Geschäftslogik festlegen.
Sie bilden den Kern der Anwendung und ändern sich nur äußerst selten.

Die Domäne der Anwendung legt bisher den Fokus auf die implementierte DSL Sprache.
Daher befinden sich die einzelnen aus den Bausteinen der Sprache transformierten Datenobjekte alle im Bereich des \emph{Domänen-Code}.
Konkret gemeint sind hiermit die Klassen aus dem Paket \texttt{it.oechsler.script.data}.

Zu diesem Schluss kommt man, wenn man die Betrachtungsweise einer klassischen Anwendung auf die in diesem Projekt definierte Kotlin DSL anwendet.
Klassische Anwendung nehmen ihre Daten aus den \emph{Adaptern} und verarbeiten diese dann mit den aus der \emph{Domänen-Schicht} hervorgehnden Datenobjekten auf der \emph{Applikations-Schicht}, um sie dann erneut an einen \emph{Adapter} weiterzugeben.\\
Nichts anderes macht die in diesem Projekt definierte DSL.
Die Eingabe erfolgt mittels der \emph{Skriptdatei} als Daten und dem \emph{Controller} der \emph{Command Line} als \emph{Adapter}.
Anschließend werden die im Skript bereitgestellten Daten auf die Datenobjekte der \emph{Domäne} unter zunahme des \emph{Applikations-Codes} (Grammatik der internen DSL) gebracht.
Das alles geschieht nun jedoch im Zuge der Ausführung des Skriptes, da es sich bei diesem um Code handelt, der aus einer internen DSL entstammt.
Letztlicht steht noch die Rückgabe über einen weiteren \emph{Adapter}.
Da die Ausgabe bisher als \emph{Dump} des erzeugten \emph{JSON} zurück in die \emph{Command Line} erfolgt, wäre der \emph{Adapter} hier der \emph{JSON} \emph{Parser} und der zugehörige \emph{Controller} für die \emph{Command Line}.
In späteren Versionen kommt an dieser Stelle allerdings eine Schnittselle zum Tragen, mit der die erarbeiteten \emph{JSON} \emph{Objekte} als \emph{Payload} an eine \emph{API} weitergeben werden.
Der eigentliche \emph{Adapter}.
Auch diese Schritte erfolgen immer noch im Kontext der Ausführung des jeweiligen Skriptes.
Allerdings sind die eigentlichen Bausteine hierfür immer innerhalb der Anwendung definiert und nicht im Skript. 
Das Skript bestimmt also lediglich, in welcher \enquote{Reihenfolge} der \emph{Applikations-Code} zum Tragen kommt.

Neben dem aus der DSL hervorgehnden \emph{Domänen-Code} gibt es darüber hinaus noch einen Anwendungsbereich, der sich mit Identitäten eines Nutzers auseinandersetzt.
Für diesen existieren ebenfalls eigene Datenobjekte in \texttt{it.oechsler.script.data}, die zum \emph{Domänen-Code} zählen.
Letztlich steht noch das Modul, welches für die Konfiguration verantwortlich ist.
Es definiert auch eigne dem \emph{Domänen-Code} zuschreiebare Objekte.
Sie finden sich im Paket \texttt{it.oechsler.script.data}

\subsection{Applikations-Schicht}
\label{subsec:appliktions_schicht}

\begin{sidewaysfigure}
    \centering

    \includeplantuml[width=\textwidth]{assets/use-cases}

    \caption{Klassendiagramm der \emph{Applikations-Schicht}}
    \label{fig:application}
\end{sidewaysfigure}

Bei der \emph{Applikations-Schicht}, oder dem \emph{Applikations-Code} handelt es sich konkret um die \emph{Use-Cases} der Anwendung.
Von bedeutung ist, dass Änderungen an dieser Schicht, keine Änderungen an der \emph{Domänen-Schicht} hervorrufen.

Für dieses Projekt spezifisch entstehen die einzelnen \emph{Use-Cases} aus den für die Sprache definierten Anforderungen. 
Indirekt bilden diese dann auch in der Implementierung die Grammatik der DSL Sprache.
Daher lassen sich alle \emph{Builder} als Teil des \emph{Applikations-Codes} fassen.
Diese finden sich im Paket \texttt{it.oechsler.script.language}.
Darüberhinaus existiert auch für die Skripte klassische Geschäftslogik, da Skripte geladen und verwendbar gemacht werden müssen.
Diese Operationen sind in einem dafür vorgesehenen \emph{Service}, dem \texttt{ScriptService}, bereitgestellt (siehe Paket \texttt{it.oechsler.script.services}).

\subsection{Adapter-Schicht}
\label{subsec:adapter_schicht}

\begin{figure}[H]
    \centering

    \includeplantuml[width=\textwidth]{assets/adapter}

    \caption{Klassendiagramm der \emph{Adapter-Schicht}}
    \label{fig:adapter}
\end{figure}

Die \emph{Adapter-Schicht} definiert sich aus den ihr namensgebenden \emph{Adaptern}.
Es geht hierbei um die abtrennung der \enquote{inneren} und \enquote{äußeren} Schicht der Anwendung.
Bei den \emph{Adaptern} kann es sich also zum Einen um eine Einheit handeln, welche Daten aus der Eingabe auf den \emph{Domänen-Code} konvertiert (also \emph{mappt}), zum Anderen aber auch um Einheiten, welche die \emph{Plugin-Schicht} konfigurieren. 
Als Beispiel wäre hier ein \emph{Controller} in einem \emph{MVC} \emph{Framework} zu nennen.

\vspace{5mm}
\begin{listing}[h]
    \kotlinfile{assets/noadapter.snug.kts}

    \caption{Beispiel für ein Skript, das die \emph{Adapter} nicht verwendet}
    \label{lst:noadapter}
\end{listing}
\vspace{5mm}

\begin{listing}[h]
    \kotlinfile{assets/adapter.snug.kts}
        
    \caption{Beispiel für ein Skript, das die \emph{Adapter} verwendet}
    \label{lst:adapter}
\end{listing}
\vspace{5mm}

Das Projekt bedient sich konkret bisher nur am Konfigurations-Einsatzzweck der \emph{Adapter-Schicht}, nicht aber an den \emph{Adaptern} zur Formatkonvertierung.
Das liegt daran, dass eine Konvertierung der Eingabedaten sich als wenig Sinnvoll ergibt.
Die DSL Sprache gibt durch die \emph{Builder} die vom \emph{Adapter} gegebene Funktion im \emph{Application-Code} als deren Operationen wieder.
Aus diesen gehen dann die konvertierten \emph{Domänen-Code} \emph{Instanzen} hervor.\\
Das Argument für die Trennung zwischen wenig veränderlichen und stark veränderlichen Schichten, was ja mit den \emph{Adaptern} versucht wird, gilt eh nicht hinreichend für die Sprache. 
Als Beispiel wäre hier eine Sprachänderung anzuführen.
Werden etwa neue Grundbausteine in die Sprache als Operationen auf den \emph{Buildern} integriert, so muss sich auch der \emph{Domänen-Code} ändern, da er um \emph{Datenobjekte} erweitert wird.
Zudem müsste man bei Änderungen am \emph{Adapter}, auch mit Sicherheit etwas an den eigentlichen Operation der Sprache verändern und damit am \emph{Applikations-Code}.
Außerdem würde das konzept von Datenobjekten als Eingabe in die DSL die gesamte Syntax nachteilig beeinflussen und diese unnötig Komplex machen (siehe hierzu die Beispiele in \Cref{lst:noadapter} und \Cref{lst:adapter}).
Hier jetzt einen \emph{Adapter} als Konvertierung einzubauen, ist schlicht unsinnig und wurde aus diesem Grund nicht implementiert.

Es existieren allerdings \emph{Adapter} für die einzelnen \emph{Befehle}, welche vom Benutzer aufgerufen werden können.
Diese \emph{Befehle} werden durch ein auf der \emph{Plugin-Schicht} angesiedeltes \emph{Framework} umgesetzt und mit einer Art minimalistischem \emph{Controller} konfiguriert.
Konfigurationen für die \emph{Befehle} finden sich in jedem Modul der Anwendung wieder, das interaktionsmöglichkeiten für den Benutzer auf der \emph{Command Line} anbietet.
Konkret finden sich die \emph{Controller} für die bisher in der Anwendung existierenden \emph{Befehle} in den folgenden Pakteken:

\begin{itemize}
    \item \texttt{it.oechsler.script.commands} -- \texttt{ApplyCommand}, \texttt{RollbackCommand}
    \item \texttt{it.oechsler.identity.commands} -- \texttt{AuthCommand}
\end{itemize}

\subsection{Plugin-Schicht}
\label{subsec:plugin_schicht}

Die \emph{Plugin-Schicht} greift nun auf die Elmente aus \emph{Adapter-Schicht} zurück und enthält konkret die \emph{Frameworks}, \emph{Tools} und \emph{Transportmittel}.

Im Falle der Anwendung wurde auf zwei dieser \emph{Plugin} \emph{Frameworks} zurückgegriffen. 
Beginnend zum Einen mit dem für die \enquote{darstellung} auf der \emph{Command Line} genutzte \texttt{CLI~KT}\footnote{\url{https://ajalt.github.io/clikt/}}.
Darüberhinaus kommt zum Anderen für die Entkopplung der einzelnen Abhänigkeiten das \emph{Dependency Injection} \emph{Framework} \texttt{Koin}\footnote{\url{https://insert-koin.io}} als \emph{Plugin} zum Einsatz.\\
Die wichtigste Gegebenheit, die beide erfüllen und warum sie sich zu den \emph{Plugins} zählen lassen, ist ihre variabilität.
Problemlos ließen sich die genannten \emph{Frameworks} austauschen, ohne dass hiervon Elemente aus dem Kern der Architektur betroffen wären.

Im hinblick auf die angesprochenen \emph{Transportmittel}, setzt die Anwendung auf \texttt{Ktor}\footnote{\url{https://ktor.io}}, um später auf externe Schnittstellen per \emph{HTTP} zugreifen zu können (derzeit nur als Abhängigkeit deklariert).
Darüber hinaus wird intern für die entkopplung von \emph{Controller} und \emph{Service / Repository} auf \texttt{kdiatR}\footnote{\url{https://github.com/enesacikoglu/kediatR}} gesetzt.
Es stellt eine Implementierung des \emph{Mediator} \emph{Entwurfsmusters} zur einfachen Verwendung dar.
Auch hier wäre es theoretisch wieder ermöglicht \enquote{relativ} einfach einen Austausch dieser vorzunehmen, ohne den Kern der Anwendung verändern zu müssen.

Letztlich gibt es jetzt noch eine extern eingebundene Validierung \texttt{valiktor}\footnote{\url{https://github.com/valiktor/valiktor}}.
Sie trifft nun aufgrund der Feststellung in \Cref{subsec:adapter_schicht} nicht mehr wirklich die Definition \emph{Plugin}.
Es wäre zwar möglich sie einfach auszutauschen, da sie an sich nur im \emph{Konstruktor} angewandt wird.
Allerdings ist es so, dass sie auf der \emph{Domänen-Schicht} zum Einsatz kommt.
Anders ist eine Validierung im Zuge der DSL und \texttt{valiktor} nicht ohne größeren Mehraufwand anwendbar.
Hier müsste man schlicht weg die Validierung von der eigentlichen \emph{Domänen-Schicht} fernhalten. 
Was wiederum durch einen \emph{Adapter} sehr einfach möglich wäre.
Gerade diese haben sich aber als eher unpraktisch für die DSL herausgestellt.
Letztlich bietet dieser Bereich langfristig definitives Verbesserungspotential was die einhaltung der \emph{Clean Architecture} angeht.
Für den Moment kann hier allerdings nicht wirklich besser vorgegangen werden.