\chapter{Ein Kapitel}
\blindtext[2]

\begin{figure}[H]
    \centering

    \missingfigure{Diagramm für irgendwas}
    
    \caption{Ein Diagramm für irgendwas}
    \label{fig:irgendwas}
\end{figure}

\section{Ein Abschnitt}
\blindtext[4]

\begin{listing}[H]
    \begin{gocode}
        package main

        import "fmt"

        func main() {
            fmt.Println("Hello, World")
        }
    \end{gocode}
    
    \caption{Code für ein \enquote{Hello World} Programm in GO (siehe \cite{go:helloworld_playground})}
    \label{alg:go_helloworld}
\end{listing}

Das in \Cref{alg:go_helloworld} dargestellte Codebeispiel kann \ac{z.B.} auch im Text referenziert werden und dazu beitragen, diesen verständlicher zu gestalten.\todo{Text hier noch vervollständigen}
In diesem Satz wurde \texthl{ein Wort} markeiert, was etwa für Korrekturen interessant ist.

\begin{table}
    \centering
    
    \settowidth\tymin{\texttt{LoadBalancedDeployment}}
    \begin{tabulary}{\textwidth}{|L|L|} 
        \hline
        \textbf{Keyword} & \textbf{Erklärung} \\
        \hline
        \texttt{Config}  & Abbildung der Konfigurationsdatei mit allen Sektionen \\
        \hline
        \texttt{AuthConfig} & Abschnitt in der \texttt{Config} spezifisch für Authentifizierung \\
        \hline
        \texttt{AuthToken} & Token mit dem sich ein Nutzer an der API authentifiziert \\
        \hline
    \end{tabulary}

    \caption{Auszug aus der \emph{Ubiquitous Language} einer Anwendung}
    \label{tab:ubiquitous_language}
\end{table}