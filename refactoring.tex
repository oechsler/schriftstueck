\section{Refactoring}
\label{sec:refactoring}

Im Sinne des \emph{Refactoring} wurden im Wesentlichen zwei konkrete dieser während der Entwicklung angewandt: \emph{Restrukturierung der Klassen in Paketen} und \emph{Umbenennung von Methoden}.

Da diese während der Durchführung nicht spezifisch betrachtet wurden, ist zum Zeitpunkt der schriftlichen Ausarbeitung nur noch eine protokollartige Wiedergabe der Ereignisse möglich.
Daher sind auch keine Klassendiagramme angegeben.

\subsection{Restruktuierung}
\label{subsec:restrukturierung}

Im Bereich der \emph{Restruktuierung} mussten einige Klassen in andere Pakete zugeteilt werden, um den entsprechenden Anwendungbereich zu sichern.
Dieses Refactoring erfolge auf Basis der Fuktionalitäten, welche von der verwendeten \emph{IDE}.

Ein konkretes Beispiel für solch ein Refactoring während der Entwicklung ist das Fehlen von Code Coverage aus den Tests.
Die \emph{IDE} hat hierfür erwartet, dass die entsprechende Struktur des Projektes mit den \texttt{package} Definitionen in den Klassen übereinstimmt.
Das war teilweise nicht der Fall.
Aufgrund dessen musste sichergestellt werden, dass die Struktur integer mit den Definitionen in den Klassen ist.

\subsection{Umbenennung}
\label{subsec:umbenennung}

Gerade bei den \texttt{Builder} Klassen und deren Methoden ist es häufig nötig gewesen Methoden umzubenennen, um sicherzustellen, dass die DSL Sprache sinnvolle und vor allem verständliche Begriffe für den Endbenutzer bereitstellt.
Dieses Refactoring wurde ebenfalls mit den von der \emph{IDE} bereitgestellten Funktionalitäten getätigt.

Konkrete Beispiele des \emph{Refactorings} lassen sich nicht reproduzieren, können aber technisch über die Historie der \emph{Versionskontrolle} eingesehen werden.
